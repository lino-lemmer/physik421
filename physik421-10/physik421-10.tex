% Für Seitenformatierung

\documentclass[DIV=15]{scrartcl}

% Zeilenumbrüche

\parindent 0pt
\parskip 6pt

% Für deutsche Buchstaben und Synthax

\usepackage[ngerman]{babel}

% Für Auflistung mit speziellen Aufzählungszeichen

\usepackage{paralist}

% zB für \del, \dif und andere Mathebefehle

\usepackage{amsmath}
\usepackage{commath}
\usepackage{amssymb}

% Für Literatur/bibliography

%\usepackage[backend=biber , style=alphabetic , hyperref=true]{biblatex}

% Für \SIunit[]{} und \num in deutschem Stil

 \usepackage[output-decimal-marker={,}]{siunitx}
 \DeclareSIUnit\clight{\ensuremath{c}}

% Schriftart und encoding

\usepackage[utf8]{inputenc}
% Bitstream charter als default
\usepackage[charter, greekuppercase=italicized]{mathdesign}
% Lato, als sans default
\renewcommand{\sfdefault}{fla}

% Für \sfrac{}{}, also inline-frac

\usepackage{xfrac}

% Für Einbinden von pdf-Grafiken

\usepackage{graphicx}

% TikZ

\usepackage{tikz}
\usetikzlibrary{arrows}

% Umfließen von Bildern

% \usepackage{floatflt}

% Für weitere Farben

\usepackage{color}

% Für Streichen von z.B. $\rightarrow$

\usepackage{centernot}

% Für Befehl \cancel{}

\usepackage{cancel}
\newcommand\ccancel[2][black]{\renewcommand\CancelColor{\color{#1}}\cancel{#2}}

% Für Links nach außen und innerhalb des Dokumentes

\usepackage{hyperref}

% Für Layout von Links

\hypersetup{
	citecolor=black,
	colorlinks=true,
	linkcolor=black,
	urlcolor=blue,
}

% Verschiedene Mathematik-Hilfen

% Manuelles taggen z.B. in align*-Umgebung

\newcommand\mantag{\stepcounter{equation}\tag{\theequation}}

\newcommand \e[1]{\cdot10^{#1}}
\newcommand\p{\partial}

\newcommand\half{\frac 12}
\newcommand\shalf{\sfrac12}

\newcommand\skp[2]{\left\langle#1,#2\right\rangle}
\newcommand\mw[1]{\left\langle#1\right\rangle}

\newcommand \ee{\mathrm e}
\newcommand \eexp[1]{\mathrm{e}^{#1}}
\newcommand \dexp[1]{\exp\left({#1}\right)}

% Trigonometrische Funktionen mit Argument in Klammern

\newcommand \dsin[1]{\sin\left({#1}\right)}
\newcommand \dcos[1]{\cos\left({#1}\right)}
\newcommand \dtan[1]{\tan\left({#1}\right)}
\newcommand \darccos[1]{\arccos\left({#1}\right)}
\newcommand \darcsin[1]{\arcsin\left({#1}\right)}
\newcommand \darctan[1]{\arctan\left({#1}\right)}

\newcommand{\ui}[1]{\int_{-\infty}^{\infty}\dif {#1}\;}

% Für fette, serifenlose Matrix

\newcommand \mat[1]{\mathbf{#1}}

% Nabla und Kombinationen von Nabla

\renewcommand\div[1]{\skp{\nabla}{#1}}
\newcommand\rot{\nabla\times}
\newcommand\grad[1]{\nabla#1}
\newcommand\laplace{\triangle}
\newcommand\dalambert{\mathop{{}\Box}\nolimits}

%Für komplexe Zahlen

\newcommand \ii{\mathrm i}
\renewcommand{\Im}{\mathop{{}\mathrm{Im}}\nolimits}
\renewcommand{\Re}{\mathop{{}\mathrm{Re}}\nolimits}

%Für Bra-Ket-Notation

\newcommand\bra[1]{\left\langle#1\right|}
\newcommand\ket[1]{\left|#1\right\rangle}
\newcommand\braket[2]{\left\langle#1\left.\vphantom{#1 #2}\right|#2\right\rangle}
\newcommand\braopket[3]{\left\langle#1\left.\vphantom{#1 #2 #3}\right|#2\left.\vphantom{#1 #2 #3}\right|#3\right\rangle}


\newcounter{thezettel}
\setcounter{thezettel}{10}
\renewcommand\thesection{\arabic{thezettel}.\arabic{section}}



\title{physik421 - Übung \arabic{thezettel}}
\author{Lino Lemmer \\ \small{l2@uni-bonn.de} \and Frederike Schrödel \and Simon Schlepphorst\\ \small{s2@uni-bonn.de}}

\begin{document}
\maketitle

\section{Spinmatrizen}

\section{Lamor Präzession}

\section{Zwei Spin-1/2 Teilchen}
\subsection{}
Es sollen die gemeinsamen Eigenzustände $\ket{S_1S_2;Sm_s}$ des Gesammtspinoperators $\hat S=\hat S_1+\hat S_2$, dessen $z$-Komponenten $\hat S_z$ und $\hat S_1^2$, $\hat S_2^2$.

Wenn man $S^2$ in der Basis $\ket{++}$, $\ket{+-}$, $\ket{-+}$ und $\ket{--}$ ausdrückt, erhält man:
\[
    S^2 = \hbar^2 \begin{pmatrix}
        2 &0 &0 &0 \\
        0 &1 &1 &0 \\
        0 &1 &1 &0 \\
        0 &0 &0 &2
    \end{pmatrix}
\]

Um an die Eigenzustände zu kommen bestimmt man zuerst die Eigenwerte. Nach einigen rechnen erhält man drei mal den Eigenwert $2\hbar$ und ein mal 0. 
Aus $\lambda = 2$ und $\lambda = 0$ erhalten wir:

\begin{align*}
\begin{pmatrix} 1 \\ 0 \\ 0 \\ 0 \end{pmatrix} &= \ket{++} = \ket{11} \\
\begin{pmatrix} 0 \\ 1 \\ 1 \\ 0 \end{pmatrix} &= \frac{1}{\sqrt 2}(\ket{+-}+\ket{-+}) = \ket{10} \\
\begin{pmatrix} 0 \\ 0 \\ 0 \\ 1 \end{pmatrix} &= \ket{--} = \ket{1-1} \\
\begin{pmatrix} 0 \\ 1 \\ -1 \\ 0 \end{pmatrix} &=\frac 1{\sqrt 2}(\ket{+-}-\ket{-+} \\
\end{align*}

Dies ist unsere neue Basis.

\subsection{}
\begin{align*}
 S^2 = \del{S_1 + S_2}^2 \implies S_1\cdot S_2 = \frac12\del{S^2 - S_1^2 - S_2^2}
\end{align*}
Die Eigenzustände $\ket{S_1 S_2; S m_S}$ sind gemeinsame Eigenzustände der Operatoren $S^2, S_z, S_1^2, S_2^2$ und damit auch zu $S_1\cdot S_2$:
\begin{align*}
 S_1S_2\ket{1m_S} &= \frac{\hbar^2}2 \del{2 - \frac34 - \frac34}\ket{lm_S} = \frac14\hbar^2\ket{1m_S}\\
 \intertext{%
  Eigenwert: $1/4 \hbar^2$}
 S_1S_2\ket{00} &= \frac{\hbar^2}2 \del{0 - \frac34 - \frac34}\ket{00} = -\frac34\hbar^2\ket{00}\\
 \intertext{%
  Eigenwert: $-(3/4)\hbar^2$}
\end{align*}

\subsection{}
$P$ ist hermitisch, da die Spinoperatoren $S_1$ und $S_2$ hermitisch sind und miteinander vertauschen. Ferner gilt:
\begin{align*}
 P\ket{1m_S} &= \del{\frac34 - \frac14} \ket{1m_S} = \ket{1m_S}\\
 P\ket{00} &= \del{\frac34 - \frac34}\ket{00} = 0\\
 \implies &P^2\ket{Sm_S} = P\ket{Sm_S}
\end{align*}
$P$ projeziert auf den Unterraum der Triplettzustände $\ket{1m_S}$.

\section{Gesamtdrehimpuls des Elektrons}

\subsection{}
Es soll gezeigt werden, dass für die Quantenzahl j nur die Werte $l + (1/2)$ und
$l - (1/2)$ möglich sind.
\begin{align*}
 \abs{l-\frac12} &\leq j \leq l + \frac12\\
 l = 0 &\implies j = \frac12\\
 l \geq 1 &\implies j = 1 + \frac12, 1 -\frac12
\end{align*}

\subsection{}
Beweis durch vollständige Induktion, beginnend mit $\ket{l + 1/2 m_j}$.\\
Für $m_j = l + 1/2$ gilt:
\begin{align*}
 \ket{l + \frac12 \;\; l + \frac12} = \ket{ll}\ket{+} &&\text{mit } \ket{+} = \ket{\frac12 \frac12}
\end{align*}
Für $m_j = l -(1/2)$ gilt dann:
\begin{align*}
 \ket{l + \frac12 \;\;l - \frac12} = \sqrt{\frac{2l}{2l + 1}} \ket{ll-1}\ket{+} + \sqrt{\frac1{2l+1}}\ket{ll}\ket{-}
\end{align*}
Vorausgesetzt die Formel ist korrekt, dann gilt für $m_j - 1$:
\begin{align*}
 J_- = &L_- + S_-\\
 J_-\ket{l+1/2m_j} = &\hbar\sqrt{\del{l+1/2+m_j}\del{l+1/2-m_j+1}} \ket{l+1/2m_j-1}\\ \\
 J_-\ket{lm_j-\frac12}\ket{+} = &\hbar\ket{lm_j - \frac12}\ket{-} + \hbar\sqrt{\del{l+m_j-\frac12}\del{l-m_j+\frac32}}\ket{lm_j-\frac32}\ket{+}\\
 J_-\ket{lm_j+\frac12}\ket{-} = &\hbar\sqrt{\del{l+m_j+\frac12}\del{l-m_j+\frac12}}\ket{lm_j}\ket{-}
 \intertext{%
  Daraus folgt:}
 \ket{l+\frac12\;\;m_j-1} = &\ket{lm_j - \frac12}\ket{-}\cbr{\sqrt{\frac1{\del{2l+1}\del{l-m_j+3/2}}} + \frac{l-m_j+1/2}{\sqrt{(2l+1)\del{l-m_j+3/2}}}}\\
 &+ \sqrt{\frac{l+m_j-1/2}{2l+1}} \ket{lm_j-\frac32}\ket{+}\\
 = &\sqrt{\frac{l-m_j+3/2}{2l+1}}\ket{lm_j-\frac12}\ket{-} + \sqrt{\frac{l+m_j-1/2}{2l+1}} \ket{lm_j-\frac32}\ket{+}\\
 \intertext{%
  Die Relation für $\ket{l+1/2m_j}$ ist damit bewiesen.}
 \intertext{%
  Für den Zustand $\ket{l-1/2m_j}$ und $m_j = l-(1/2)$ gilt dann:}
 \ket{l-\frac12 \;\; l -\frac12} = &\sqrt{\frac1{2l+1}}\ket{ll-1}\ket{+} - \sqrt{\frac{2l}{2l+1}}\ket{ll}\ket{-}\\
 \intertext{%
  Wir schließen von $m_j$ auf $m_j-1$}
 J_-\ket{l-\frac12m_j} = &\hbar\sqrt{\del{l-\frac12+m_j}\del{l+\frac12-m_j}}\ket{l-\frac12\;\;m_j-1}\\
 \\
 J_-\ket{lm_j - \frac12}\ket{+} = &\hbar\ket{lm_j-\frac12}\ket{-} + \hbar\sqrt{\del{l+m_j-\frac12}\del{l-m_j+\frac32}}\ket{lm_j-\frac32}\ket{+}\\
 J_-\ket{lm_j+\frac12}\ket{-} = &\hbar\sqrt{\del{l+m_j+\frac12}\del{l-m_j+\frac12}}\ket{lm_j-\frac12}\ket{-}\\
 \intertext{%
  Daraus folgt wieder:}
 \ket{l-\frac12\;\; m_j-1} = &\sqrt{\frac1{(2l+1)\del{l+m_j-1/2}}}\ket{lm_j-\frac12}\ket{-} + \sqrt{\frac{l-m_j+3/2}{2l+1}}\ket{lm_j-\frac32}\ket{+}\\
 &- \sqrt{\frac{\del{l+m_j+1/2}^2}{\del{2l+1}\del{l+m_j-1/2}}}\ket{lm_j-\frac12}\ket{-}\\
 = &\sqrt{\frac{l-m_j+3/2}{2l+1}}\ket{lm_j-\frac32}\ket{+} - \sqrt{\frac{l+m_j-1/2}{2l+1}}\ket{lm_j-\frac12}\ket{-}
\end{align*}
Damit ist der Beweis vollständig.


\end{document}
