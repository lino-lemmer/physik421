% Für Seitenformatierung

\documentclass[DIV=15]{scrartcl}

% Zeilenumbrüche

\parindent 0pt
\parskip 6pt

% Für deutsche Buchstaben und Synthax

\usepackage[ngerman]{babel}

% Für Auflistung mit speziellen Aufzählungszeichen

\usepackage{paralist}

% zB für \del, \dif und andere Mathebefehle

\usepackage{amsmath}
\usepackage{commath}
\usepackage{amssymb}

% Für Literatur/bibliography

%\usepackage[backend=biber , style=alphabetic , hyperref=true]{biblatex}

% Für \SIunit[]{} und \num in deutschem Stil

 \usepackage[output-decimal-marker={,}]{siunitx}
 \DeclareSIUnit\clight{\ensuremath{c}}

% Schriftart und encoding

\usepackage[utf8]{inputenc}
% Bitstream charter als default
\usepackage[charter, greekuppercase=italicized]{mathdesign}
% Lato, als sans default
\renewcommand{\sfdefault}{fla}

% Für \sfrac{}{}, also inline-frac

\usepackage{xfrac}

% Für Einbinden von pdf-Grafiken

\usepackage{graphicx}

% TikZ

\usepackage{tikz}
\usetikzlibrary{arrows}

% Umfließen von Bildern

% \usepackage{floatflt}

% Für weitere Farben

\usepackage{color}

% Für Streichen von z.B. $\rightarrow$

\usepackage{centernot}

% Für Befehl \cancel{}

\usepackage{cancel}
\newcommand\ccancel[2][black]{\renewcommand\CancelColor{\color{#1}}\cancel{#2}}

% Für Links nach außen und innerhalb des Dokumentes

\usepackage{hyperref}

% Für Layout von Links

\hypersetup{
	citecolor=black,
	colorlinks=true,
	linkcolor=black,
	urlcolor=blue,
}

% Verschiedene Mathematik-Hilfen

% Manuelles taggen z.B. in align*-Umgebung

\newcommand\mantag{\stepcounter{equation}\tag{\theequation}}

\newcommand \e[1]{\cdot10^{#1}}
\newcommand\p{\partial}

\newcommand\half{\frac 12}
\newcommand\shalf{\sfrac12}

\newcommand\skp[2]{\left\langle#1,#2\right\rangle}
\newcommand\mw[1]{\left\langle#1\right\rangle}

\newcommand \ee{\mathrm e}
\newcommand \eexp[1]{\mathrm{e}^{#1}}
\newcommand \dexp[1]{\exp\left({#1}\right)}

% Trigonometrische Funktionen mit Argument in Klammern

\newcommand \dsin[1]{\sin\left({#1}\right)}
\newcommand \dcos[1]{\cos\left({#1}\right)}
\newcommand \dtan[1]{\tan\left({#1}\right)}
\newcommand \darccos[1]{\arccos\left({#1}\right)}
\newcommand \darcsin[1]{\arcsin\left({#1}\right)}
\newcommand \darctan[1]{\arctan\left({#1}\right)}

\newcommand{\ui}[1]{\int_{-\infty}^{\infty}\dif {#1}\;}

% Für fette, serifenlose Matrix

\newcommand \mat[1]{\mathbf{#1}}

% Nabla und Kombinationen von Nabla

\renewcommand\div[1]{\skp{\nabla}{#1}}
\newcommand\rot{\nabla\times}
\newcommand\grad[1]{\nabla#1}
\newcommand\laplace{\triangle}
\newcommand\dalambert{\mathop{{}\Box}\nolimits}

%Für komplexe Zahlen

\newcommand \ii{\mathrm i}
\renewcommand{\Im}{\mathop{{}\mathrm{Im}}\nolimits}
\renewcommand{\Re}{\mathop{{}\mathrm{Re}}\nolimits}

%Für Bra-Ket-Notation

\newcommand\bra[1]{\left\langle#1\right|}
\newcommand\ket[1]{\left|#1\right\rangle}
\newcommand\braket[2]{\left\langle#1\left.\vphantom{#1 #2}\right|#2\right\rangle}
\newcommand\braopket[3]{\left\langle#1\left.\vphantom{#1 #2 #3}\right|#2\left.\vphantom{#1 #2 #3}\right|#3\right\rangle}


\newcounter{thezettel}
\setcounter{thezettel}{9}
\renewcommand\thesection{\arabic{thezettel}.\arabic{section}}



\title{physik421 - Übung \arabic{thezettel}}
\author{Lino Lemmer \\ \small{l2@uni-bonn.de} \and Frederike Schrödel \and Simon Schlepphorst\\ \small{s2@uni-bonn.de}}

\begin{document}
\maketitle

\section{Drehmatrix}

\subsection{}

Eine Drehmatrix erfüllt folgende Eigenschaften:
\begin{itemize}
    \item
        $\det(\mat{M}) = 1$
    \item
        $\mathbf{M}\cdot\mathbf{M}^\text{T} = \mathbb{1}$
\end{itemize}

Da für die gegebene Matrix jedoch gilt
\[
    \det(\mat D) = \frac1{\sqrt{2}},
\]
in den folgenden Teilaufgaben jedoch davon ausgegangen wird, die Matrix sei eine Drehmatrix, gehe ich von folgender Matrix aus:
\[
    \mat D = \frac{1}{\sqrt{2}}
    \begin{pmatrix}
        -1 & 0 & -1 \\
        0 & \sqrt2 & 0 \\
        1 & 0 & -1
    \end{pmatrix}.
\]
Für diese prüfe ich die oben genannten Eigenschaften.
\begin{align*}
    \det (\mat D) &= \del{\frac1{\sqrt{2}}}^3 \cdot
    \begin{vmatrix}
        -1 & 0 & -1 \\
        0 & \sqrt2 & 0 \\
        1 & 0 & -1
    \end{vmatrix} \\
    &= \del{\frac1{\sqrt{2}}}^3 \cdot\del{ - 0 \cdot 
    \begin{vmatrix}
        0 & 0 \\
        1 & -1
    \end{vmatrix} + \sqrt2 \cdot \begin{vmatrix}
        -1 & -1 \\
        1 & -1
    \end{vmatrix} - 0 \cdot \begin{vmatrix}
        -1 & -1 \\
        0 & 0
    \end{vmatrix}} \\
    &= \del{\frac1{\sqrt{2}}}^2 \cdot
    \begin{vmatrix}
    -1 & -1 \\
    1 & -1
    \end{vmatrix} \\
    &= \half \del{(-1)\cdot(-1)-1\cdot(-1)} \\
    &= 1
    \intertext{%
        Damit wäre die erste Bedingung erfüllt.
    }
    \mat D \cdot \mat D^\text{T} &= \half
    \begin{pmatrix}
        -1 & 0 & -1 \\
        0 & \sqrt2 & 0 \\
        1 & 0 & -1
    \end{pmatrix} \cdot
    \begin{pmatrix}
        -1 & 0 & 1 \\
        0 & \sqrt2 & 0 \\
        -1 & 0 & -1
    \end{pmatrix} \\
    &= \half
    \begin{pmatrix}
        1+1 & 0 & 1-1 \\
        0 & 2 & 0 \\
        1-1 & 0 1+1
    \end{pmatrix} \\
    &= \mathbb 1
    \intertext{%
        Beide Bedingungen sind erfüllt, es handelt sich also tatsächlich um eine Drehmatrix. Um nun herauszufinden, was sie für eine Drehung verursacht, lassen wir sie auf einen Vektor $\vec x$ wirken.
    }
    \mat D\cdot \vec x &= \frac 1{\sqrt2}
    \begin{pmatrix}
        -1 & 0 & -1 \\
        0 & \sqrt2 & 0 \\
        1 & 0 & -1
    \end{pmatrix} \cdot
    \begin{pmatrix}
        x_1 \\
        x_2 \\
        x_3
    \end{pmatrix}  \\
    &= \begin{pmatrix}
    \frac1{\sqrt2}\del{-x_1-x_3} \\
    x_2 \\
    \frac1{\sqrt2}\del{x_1-x_3} \mantag \label{eq:gedreht}
\end{pmatrix}.
\end{align*}
Da die zweite Komponente des Vektors durch die Drehung nicht verändert wird, handelt es sich ganz offensichtlich um eine Drehung um die $y$-Achse.

\subsection{}

Mit \eqref{eq:gedreht} erhalten wir für die beiden Vektoren
\begin{align*}
    \vec a' &= \mat D \cdot \del{0,-2,1}^\text{T} \\
            &= \begin{pmatrix}
    \frac{1}{\sqrt{2}}\cdot\del{-0 - 1} \\
    -2 \\
    \frac{1}{\sqrt{2}}\cdot\del{0 - 1}
    \end{pmatrix} \\
    &= \begin{pmatrix}
    -\frac1{\sqrt2} \\
    -2 \\
    -\frac1{\sqrt2}
\end{pmatrix} \\
    \vec b' &= \mat D \cdot \del{3,5,-4}^\text{T} \\
            &= \begin{pmatrix}
    \frac1{\sqrt2}\cdot\del{-3+4} \\
        5 \\
        \frac1{\sqrt{2}}\del{3+4}
    \end{pmatrix} \\
    &= \begin{pmatrix}
    \frac1{\sqrt2} \\
    5 \\
    \frac7{\sqrt2}
\end{pmatrix}
\end{align*}

\subsection{}

Da gilt 
\begin{align*}
    \skp{\vec a}{\vec b} &= \abs a \cdot \abs b \dcos{\alpha},
    \intertext{%
        mit dem Winkel zwischen den Vektoren $\alpha$, und sowohl die Beträge der Vektoren, als auch der Winkel zwischen ihnen bei einer Drehung konstant bleiben, sollte das Skalarprodukt vor und nach der Drehung gleich sein. Dies Prüfen wir nun.
    }
    \skp{\vec a}{\vec b} &= \skp{\begin{pmatrix}
    0 \\
    -2 \\
    1
    \end{pmatrix}}{\begin{pmatrix}
        3 \\
        5 \\
        -4
    \end{pmatrix}} \\
    &= 0 - 10 -4 \\
    &= -14 \\
    \skp{\vec a'}{\vec b'} &= \skp{\begin{pmatrix}
    -\frac1{\sqrt2} \\
    -2 \\
    -\frac1{\sqrt2}
    \end{pmatrix}}{\begin{pmatrix}
    \frac1{\sqrt2} \\
    5 \\
    \frac7{\sqrt2}
    \end{pmatrix}} \\
    &= -\half - 10 - \frac72 \\
    &= -14
\end{align*}
Unsere Vermutung ist somit bestätigt.

\subsection{}

Wie in der vorherigen Teilaufgabe schon erwähnt, ändern sich die Beträge der Vektoren nicht unter Drehung. Auch dies überprüfen wir noch.

\begin{align*}
    \abs{\vec a} &= \sqrt{0^2+(-2)^2+1^2} = \sqrt5 \\
    \abs{\vec a'} &= \sqrt{\del{-\frac1{\sqrt2}}^2+(-2)^2+\del{-\frac1{\sqrt2}}^2} = \sqrt5 \\
    \abs{\vec b} &= \sqrt{3^2+5^2+(-4)^2} = \sqrt{50} \\
    \abs{\vec b} &= \sqrt{\del{\frac1{\sqrt2}}^2+5^2+\del{\frac7{\sqrt2}}^2} = 50
\end{align*}
Dies stimmt mit der Vermutung überein.

\section{Quantenmechanischer Drehimpulsoperator}

\subsection{}

Es ist zu zeigen, dass der Drehimpulsoperator $\hat L = \hat X \times \hat P$ hermitesch ist, also dass gilt $\hat L^\dagger = \hat L$.
\begin{align*}
    \hat L_i &= \sum_{j=1}^3\sum_{k=1}^3 \epsilon_{ijk}\hat X_j \hat P_k 
    \intertext{%
        Da nur für $j\neq k$ das Levi-Civita-Symbol von null verschieden ist, kommutieren die entsprechenden Komponenten, und man kann schreiben
    }
             &= \sum_{j=1}^3\sum_{k=1}^3 \epsilon_{ijk}\hat P_k \hat X_j \\
             &= \sum_{j=1}^3\sum_{k=1}^3 \epsilon_{ijk}\hat P_k^\dagger \hat X_j^\dagger \\
             &= \sum_{j=1}^3\sum_{k=1}^3 \epsilon_{ijk}\del{\hat X_j\hat P_k}^\dagger \\
             &= \hat L_i^\dagger.
\end{align*}

\subsection{}
\label{ssec:L_iL_j}

Die Kommutatorrelation $\sbr{\hat L_i, \hat L_j} =
\ii\hbar\sum_{k=1}^3\epsilon_{ijk}\hat L_k$ soll bestätigt werden. Um den
Beweis übersichtlicher zu gestalten, verwenden wir die Einsteinsche
Summenkonvention, nach der über mehrfach auftauchende Indizes summiert wird.
Zudem lassen wir die Hüte auf den Operatoren weg.
\begin{align*}
    \sbr{L_i, L_j} &= \sbr{\epsilon_{ikl}X_kP_l, \epsilon_{jmn}X_mP_n} \\
                   &= \epsilon_{ikl}X_kP_l\epsilon_{jmn}X_mP_n - \epsilon_{jmn}X_mP_n\epsilon_{ikl}X_kP_l \\
                   &=  \epsilon_{ikl}\epsilon_{jmn}\del{X_kP_lX_mP_n - X_mP_nX_kP_l}. 
    \intertext{%
        In Ortskoordinaten lautet der Impulsoperator $P_n = \frac\hbar\ii\p_n$,
        wobei $\p_n = \dpd{}{X_n}$. Setzen wir dies ein erhalten wir
    }
    &= -\hbar^2\epsilon_{ikl}\epsilon_{jmn}\del{X_k\p_l\del{X_m\p_n} -
X_m\p_n\del{X_k\p_l}} \\
    &= -\hbar^2\epsilon_{ikl}\epsilon_{jmn}\del{X_k\del{\p_lX_m}\p_n +
X_kX_m\p_l\p_n - X_m\del{\p_nX_k}\p_l - X_mX_k\p_n\p_l}.
    \intertext{%
       Da nun die $X$-Komponenten vertauschen und ebenso die Ableitungen,
       erhalten wir
   }
    &= -\hbar^2\epsilon_{ikl}\epsilon_{jmn}
   \del{X_k\underbrace{\del{\p_lX_m}}_{=\delta_{lm}}\p_n -
   X_m\underbrace{\del{\p_nX_k}}_{=\delta_{nk}}\p_l } \\
   &= -\hbar^2\epsilon_{ikl}\epsilon_{jln}X_k\p_n + \hbar^2\epsilon_{ikl}\epsilon_{jmk}X_m\p_l \\
   &= \hbar^2\epsilon_{ikl}\epsilon_{jnl}X_k\p_n - \hbar^2\epsilon_{ilk}\epsilon_{jmk}X_m\p_l \\
   &= \hbar^2X_k\p_n\del{\delta_{ij}\delta_{kn} - \delta_{in}\delta_{jk}} - \hbar^2X_m\p_l\del{\delta_{ij}\delta_{lm} - \delta_{im}\delta_{jl}}.
   \intertext{%
       Nun benennen wir die Indizes $k$ und $n$ um: $k \to m$ und $n\to l$.
   }
   &= \hbar^2X_m\p_l\del{\delta_{ij}\delta_{ml} - \delta_{il}\delta_{jm}} - \hbar^2X_m\p_l\del{\delta_{ij}\delta_{lm} - \delta_{im}\delta_{jl}} \\
   &= -\hbar^2 \del{\delta_{il}\delta_{jm}-\delta_{im}\delta_{jl}}X_m\p_l \\
   &= -\hbar^2 \epsilon_{ijk}\epsilon_{lmk}X_m\p_l \\
   &= -\ii\hbar \epsilon_{ijk}\epsilon_{lmk}X_mP_l \\
   &= \ii\hbar \epsilon_{ijk}\epsilon_{kml}X_mP_l \\
   &= \ii\hbar \epsilon_{ijk} L_k
\end{align*}
Die Kommutatorrelation ist somit bewiesen.

\subsection{}

Zunächst ist zu zeigen, dass $\sbr{L_i,X_j} = \ii\hbar\epsilon_{ijk}X_k$ gilt.
\begin{align*}
    \sbr{L_i,X_j} &= \epsilon_{ikl}X_kP_lX_j-X_j\epsilon_{ikl}X_kP_l \\
                  &= \frac\hbar\ii\epsilon_{ikl}\del{X_k\p_lX_j-X_jX_k\p_l} \\
                  &= \frac\hbar\ii\epsilon_{ikl}\del{X_k\p_l\del{X_j}+X_kX_j\p_l-X_jX_k\p_l}.
    \intertext{%
        Mit $\p_l\del{X_j} = \delta_{lj}$ und $\sbr{X_k,X_j}=0$ erhalten wir
    }
    &= \frac\hbar\ii\epsilon_{ikj}X_k \\
    &= \ii\hbar \epsilon_{ijk}X_k. \tag*{$\square$}
    \intertext{%
        Nun soll gezeigt werden, dass $\sbr{L_i,P_j}=\ii\hbar\epsilon_{ijk}P_k$ gilt.
    }
    \sbr{L_i,P_j} &= \epsilon_{ikl}X_kP_lP_j - P_j\epsilon_{ikl}X_kP_l \\
                  &= -\hbar^2\epsilon_{ikl}\del{X_k\p_l\p_j - \p_jX_k\p_l} \\
                  &= -\hbar^2\epsilon_{ikl}\del{X_k\p_l\p_j - \p_j\del{X_k}\p_l - X_k\p_j\p_l}.
    \intertext{%
        Mit $\sbr{\p_l,\p_j} = 0$ und $\p_j\del{X_k} = \delta_{jk}$ erhalten wir
    }
    &= \hbar^2\epsilon_{ijl}\p_l \\
    &= -\ii\hbar\epsilon_{ijl}P_l.\tag*{$\cancel{\square}$}
    \intertext{%
        Dies ist irgendwie nicht ganz richtig\dots\newline
        Nun wollen wir die Kommutatorrelation $\sbr{L,X^2}=0$ bestätigen. Da
        der Drehimpulsoperator vektorwertig ist, $X^2$ jedoch ein Skalar, muss
        für jede Komponente von $L$ gelten $\sbr{L_i,X_1^2+X_2^2+X_3^2}=0$. Dafür berechnen wir zunächst den Kommutator von $L_i$ mit $X_j^2$.
    }
    \sbr{L_i,X_j^2} &= \epsilon_{ikl} \del{X_kP_lX_j^2-X_j^2PX_kP_l} \\
                    &= \frac\hbar\ii\epsilon_{ikl} \del{X_k\p_lX_j^2-X_j^2X_k\p_l} \\
                    &= \frac\hbar\ii\epsilon_{ikl} \del{X_k\p_l\del{X_j^2}+X_kX_j^2\p_l - X_j^2X_k\p_l}.
    \intertext{%
        Mit $\sbr{X_k,X_j^2}=0$ und $\p_l\del{X_j^2}=2X_j\delta_{jl}$ erhalten wir
    }
    &= -\ii\hbar\epsilon_{ikj}2X_kX_j \\
    &= 2\ii\hbar\epsilon_{ijk}X_kX_j.
    \intertext{%
        Nun schauen wir uns den gesamten Kommutator an.
    }
    \sbr{L_i,X_1^2+X_2^2+X_3^2} &= \sbr{L_i, X_1^2} + \sbr{L_i, X_2^2} + \sbr{L_i, X_3^2} \\
                                &= 2\ii\del{\epsilon_{i1k}X_kX_1 + \epsilon_{i2k}X_kX_2 + \epsilon_{i3k}X_kX_3}.
    \intertext{%
        Sein nun o.B.d.A $i=1$. Dann gilt
    }
    &= 2\ii\del{\underbrace{\epsilon_{11k}X_kX_1}_{=0} + \underbrace{\epsilon_{12k}X_kX_2}_{\implies k=3} + \underbrace{\epsilon_{13k}X_kX_2}_{\implies k=2}} \\
    &= 2\ii\del{\epsilon_{123}X_3X_2 + \epsilon_{132}X_2X_3} \\
    &= 2\ii\del{X_3X_2 - X_2X_3} \\
    &= 0. \tag*{$\square$}
    \intertext{%
        Da dies für jede Komponente von $L$ gilt, verschwindet auch der Kommutator $\sbr{L,X^2}$.\newline
        Als nächstes zeigen wir $\sbr{L,P^2}=0$. Auch hier fangen wir Komponentenweise an:
    }
    \sbr{L_i,P_j^2} &= \epsilon_{ikl}\del{X_kP_lP_j^2 - P_j^2X_kP_l} \\
                    &= \ii\hbar^3\epsilon_{ikl}\del{X_k\p_l\p_j^2 - \p_j^2X_k\p_l} \\
                    &= \ii\hbar^3\epsilon_{ikl}\del{X_k\p_l\p_j^2 - \p_j\del{\delta_{jk}\p_l + X_k\p_j\p_l}} \\
                    &= \ii\hbar^3\epsilon_{ikl}\del{X_k\p_l\p_j^2 - \p_j\del{X_k}\p_j\p_l - X_k\p_j^2\p_l} .
    \intertext{%
        Da gilt $\sbr{\p_l,\p_j^2}=0$ erhalten wir
    }
    &= -\ii\hbar^3\epsilon_{ilj}\p_j\p_l.
    \intertext{%
        Dann schauen wir uns wieder den gesamten Kommutator an.
    }
    \sbr{L_i,P_1^2+P_2^2+P_3^2} &= \sbr{L_i,P_1^2} + \sbr{L_i,P_2^2} + \sbr{L_i,P_3^2} \\
                                &= -\ii\hbar^3\del{\epsilon_{il1}\p_1\p_l + \epsilon_{il2}\p_2\p_l + \epsilon_{il3}\p_3\p_l}.
    \intertext{%
        Sei nun o.B.d.A. $i=1$.
    }
    &= -\ii\hbar^3\del{\underbrace{\epsilon_{1l1}\p_1\p_l}_{=0} + \underbrace{\epsilon_{1l2}\p_2\p_l}_{\implies l=3} + \underbrace{\epsilon_{1l3}\p_3\p_l}_{\implies l=2}} \\
    &= -\ii\hbar^3\del{\epsilon_{132}\p_2\p_3 + \epsilon_{123}\p_3\p_2} \\
    &= -\ii\hbar^3\del{-\p_2\p_3 + \p_3\p_2} \\
    &= 0. \tag*{$\square$}
    \intertext{%
        Da dies für jede Komponente von $L$ gilt, verschwindet auch der Kommutator $\sbr{L,P^2}$.\newline
        Zu guter letzt bestätigen wir noch, dass $\sbr{L,\skp{X}{P}}=0$ gilt. Wir gehen hier analog zu den anderen Skalarprodukten vor.
    }
    \sbr{L_i,X_jP_j} &= \epsilon_{ikl}\del{X_kP_lX_jP_j-X_jP_jX_kP_l} \\
                     &= -\ii\hbar^2\epsilon_{ikl}\del{X_k\p_lX_j\p_j-X_j\p_jX_k\p_l} \\
                     &= -\ii\hbar^2\epsilon_{ikl}\del{X_k\p_l\del{X_j}\p_j+X_kX_j\p_l\p_j - X_j\p_j\del{X_k}\p_l - X_jX_k\p_j\p_l}.
    \intertext{%
        Mit $\sbr{X_i,X_j}=\sbr{P_i,P_j}=0$ und den oben schon mehrfach verwendeten Kronecker-Deltas erhalten wir
    }
    &= -\ii\hbar^2\del{\epsilon_{ikj}X_k\p_j - \epsilon_{ijl}X_j\p_l} \\
    \sbr{L_i,X_1P_1 + X_2P_2 + X_3P_3} &= -\ii\hbar^2\del{\epsilon_{ik1}X_k\p_1-\epsilon_{i1l}X_1\p_l + \epsilon_{ik2}X_k\p_2-\epsilon_{i2l}X_2\p_l + \epsilon_{ik3}X_k\p_3-\epsilon_{i3l}X_3\p_l}.
    \intertext{%
        Nun setzen wir wieder o.B.d.A. $i=1$. Wir erhalten
    }
    &= -\ii\hbar^2\del{\underbrace{\epsilon_{1k1}X_k\p_1 -
    \epsilon_{11l}X_1\p_l}_{=0} + \underbrace{\epsilon_{1k2}X_k\p_2 -
\epsilon_{12l}X_2\p_l}_{\implies k=l=3} + \underbrace{\epsilon_{1k3}X_k\p_3 -
\epsilon_{13l}X_3\p_l}_{\implies k=l=2}} \\
    &= -\ii\hbar^2\del{\epsilon_{132}X_3\p_2 - \epsilon_{123}X_2\p_3 +
    \epsilon_{123}X_2\p_3 - \epsilon_{132}X_3\p_2} \\
    &= 0. \tag*{$\square$}
\end{align*}

Auch hier gilt, da der Kommutator für jede Komponente von $L$ verschwindet,
gilt dies auch für $L$.

\subsection{}

Es ist zu zeigen, dass aus $\sbr{L_i,A}=\sbr{L_j,A}=0$ folgt $\sbr{L_k,A}=0$.
Aus Abschnitt~\ref{ssec:L_iL_j} folgt
    \begin{align*}
        \sbr{L_k,A} &= \frac1{\ii\hbar}\epsilon_{ijk} \sbr{\sbr{L_i,L_j},A} \\
                    &=\frac1{\ii\hbar}\epsilon_{ijk} \del{L_iL_jA - L_jL_iA - AL_iL_j + AL_jL_i} .
        \intertext{%
            Wegen der vorausgesetzten Vertauschbarkeit können wir in jedem Summanden den Operator $A$ zwischen die Drehimpulskomponenten bringen:
        }
                    &= \frac1{\ii\hbar}\epsilon_{ijk} \del{L_iAL_j - L_jAL_i - L_iAL_j + L_jAL_i} \\
        &= 0. \tag*{$\square$}
    \end{align*}
\end{document}
